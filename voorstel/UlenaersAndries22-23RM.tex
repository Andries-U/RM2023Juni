%==============================================================================
% Voorbeeld hogent-article: onderzoeksvoorstel bachproef
%==============================================================================

\documentclass{hogent-article}

\usepackage{lipsum} % Voor vultekst
% Invoegen bibliografiebestand
\addbibresource{references.bib}

% Informatie over de opleiding, het vak en soort opdracht
\studyprogramme{Professionele bachelor toegepaste informatica}
\course{Research Methods}
\assignmenttype{Paper: Onderzoeksvoorstel}
\academicyear{2022-2023} 

% TODO (fase 1): Werktitel
\title{Werktitel van het voorstel}


\author{Andries Ulenaers}
\email{andries.ulenaers@student.hogent.be}



\projectrepo{https://github.com/Andries-U/RM2023Juni}

% Binnen welke specialisatierichting uit 3TI situeert dit onderzoek zich?
% Kies uit deze lijst:
%
% - Mobile \& Enterprise development
% - AI \& Data Engineering
% - Functional \& Business Analysis
% - System \& Network Administrator
% - Mainframe Expert
% - Als het onderzoek niet past binnen een van deze domeinen specifieer je deze
%   zelf
%
\specialisation{Nuclear medicine \& Enterprise development}
% Geef hier enkele sleutelwoorden die je onderwerp beschrijven
\keywords{Development, Nuclear medicine, Image recognition, Process automation}

\begin{document}

\begin{abstract}
  In deze bachelorproef gaan we de werkwijze voor het analyseren van controletesten met als doel om de nauwkeurigheid en betrouwbaarheid van de PET-beelden van PET-CT scanners te valideren. Dit door middel van objectherkenning op de beelden. Hiermee zal het handmatig aftekenen van het object op de beelden geautomatiseerd worden, waardoor de efficiëntie en nauwkeurigheid verbeteren. Ook zal er hierdoor minder precisie nodig zijn bij het uitvoeren van de metingen omdat het object niet op exact dezelfde plaats moet staan bij elke slice.


  Vanwege wereldwijd versplinterde wetgeving en een beperkt aantal producenten van PET-CT scanners. Zijn controle organismes verplicht om eigen software te ontwikkelen voor het controleren van deze toestellen. De huidige werkwijze is nog te handmatig en hierdoor tijdrovend en duurt ongeveer 1 uur. Door wat IT kennis toe te voegen aan dit vakgebied kan deze tijd bespaart worden.

  Ik heb in deze scriptie de plugins testerom en testerdom vergeleken met handmatig werk op nauwkeurigheid en betrouwbaarheid. Deze plugins zijn gekozen vanwege hun populariteit en hun compatibiliteit binnen de bestaande software-suite. Met beide plugins is dan een proof-of-concept gebouwd die naadloos de resultaten doorgeeft aan Excel voor de verdere rapportage binnen het bestaande proces voor rapportering. Hiernaast genereert deze ook een afzonderlijk rapport voor latere manuele controle indien er twijfel ontstaat over de resultaten.
  
  Uit het onderzoek blijkt dat de objectherkenning en de implementatie voldoen aan de eisen voor correctheid en zeker van proof-of-concept over kunnen gaan tot productiewaardige software. Ook geeft de proof-of-concept genoeg vertrouwen dat de gewonnen tijd, kennis en nauwkeurigheid de kosten voor de ontwikkeling overstijgen.
  
\end{abstract}

\tableofcontents

\bigskip

% TODO: Neem je dit jaar ook de bachelorproef op? Haal dan de tekst hieronder
% uit commentaar en pas het aan.

%\paragraph{Opmerking}

% Ik neem dit jaar ook de bachelorproef op. De inhoud van dit onderzoeksvoorstel dient ook als het onderwerpvoor mijn bachelorproef. Mijn promotor is (Mr./Mevr.) X.\ Familienaam.

% Beschrijf de eventuele verschillen en/of verbeteringen in dit document t.o.v.\ jouw onderzoeksvoorstel dat je ingediend hebt voor de bachelorproef.

\section{Inleiding}%
\label{sec:inleiding}

% TODO: (fase 1) introduceer je gekozen onderwerp, formuleer de onderzoeksvraag en deelvragen. Wat is de doelstelling (is die S.M.A.R.T.?), wat zal het resultaat zijn van het onderzoek (een Proof-of-Concept, een prototype, een advies, ...)? Waarom is het nuttig om dit onderwerp te onderzoeken?

Vanuit een belgisch controle-organisme gespecialiseerd in nucleaire geneeskunde kwam de vraag om een add-on of alleenstaande tool te ontwikkelen. Deze tool moet het handmatig tekenwerk van hun fysici verlichten. De tool dient voor het controleren van de correcte werking van PET-CT scanners. In de huidige werkwijze wordt er te veel vertrouwen gelegd bij de interne testen van de verschillende fabrikanten van de toestellen. Deze interne testen zijn volledig afgeschermd. Hierdoor weten de fysici niet of de toestellen wel voldoen aan de gestelde eisen en zijn de criteria ook verschillend per fabrikant. 

Momenteel hebben ze al een eigen gestandardiseerde methode. Enkel moet hierbij op alle slices van de DICOM beelden verkregen bij de testscan de outline van het testobject getekend worden alsook de omzetfactor van grijswaardes naar activiteit. Het doel van de tool is om de beelden in te laden, dan via objectherkenning zelf het testobject af te tekenen om uiteindelijk via enkele berekeningen aan de hand van de grijswaarde een omzetfactor tussen meting en werkelijkheid te bekomen. De uitkomst moet dan automatisch in de excel-file van het keuringsverslag komen. Ook wensen ze de ruwe data, liefst al afgetekend, te bewaren voor manuele hercontrole indien gewenst. 

In dit onderzoek word onderzocht of de beeldherkinning mogelijk is met ImageJ, het programma dat al gebruikt wordt binnen de firma, of dat er mogelijks met andere technologie gewerkt moet worden. Hierna moet gevalideerd worden dat de beeldherkenning betrouwbaar genoeg is om in productie te nemen en op te vertrouwen als keurresultaat.

Indien de proof-of-concept voldoet kan de firma veel werkuren besparen. Hierdoor zal er meer tijd beschikbaar zijn om verbeteringsprojecten op te starten in samenwerking met de verschillende diensten nucleaire geneeskunde waarmee ze samenwerken. Dit zal zowel patienten als personeel ten goede komen.


\section{Literatuurstudie}%
\label{sec:literatuurstudie}

% TODO: (fase 4) schrijf de literatuurstudie uit en gebruik waar gepast referenties naar de vakliteratuur.

% Refereren naar de literatuur kan met:
% \autocite{BIBTEXKEY} -> (Auteur, jaartal)
% \textcite{BIBTEXKEY} -> Auteur (jaartal)
Voorbeeld van een referentie waar de auteursnaam geen onderdeel van de zin is~\autocite{Moore2002}.

\lipsum[4-9]

\section{Methodologie}%
\label{sec:methodologie}

% TODO: (fase 5) beschrijf in detail in welke fasen je onderzoek uiteenvalt, hoe lang elke fase duurt en wat het concrete resultaat van elke fase is. Welke onderzoekstechniek ga je toepassen om elk van je onderzoeksvragen te beantwoorden? Gebruik je hiervoor experimenten, vragenlijsten, simulaties? Je beschrijft ook al welke tools je denkt hiervoor te gebruiken of te ontwikkelen.

\lipsum[10-12]

\section{Verwachte resultaten}%
\label{sec:verwachte-resultaten}

% TODO: (fase 6) beschrijf wat je verwacht uit je onderzoek en waarom (bv. volgens je literatuuronderzoek is softwarepakket A het meest gebruikte en denk je dat het voor deze casus ook het meest geschikt zal zijn). Natuurlijk kan je niet in de toekomst kijken en mag je geen alternatieve mogelijkheden uitsluiten. In de praktijk gebeurt het ook vaak dat een onderzoek tot verrassende resultaten leidt, dat maakt het proces nog interessanter!

\lipsum[14-18]

\section{Discussie, conclusie}%
\label{sec:discussie-conclusie}

\lipsum[19-21]

%------------------------------------------------------------------------------
% Referentielijst
%------------------------------------------------------------------------------
% TODO: (fase 4) de gerefereerde werken moeten in BibTeX-bestand
% bibliografie.bib voorkomen. Gebruik JabRef om je bibliografie bij te
% houden.

\printbibliography[heading=bibintoc]

\end{document}